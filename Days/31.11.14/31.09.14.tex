
  \begin{enumerate}
    \item Дата собрания : 31.09.14
    \item Цель:
    \begin{itemize}
      \item Установить мебельные рейки боком к земле для того, чтобы они выдерживали большую нагрузку. Мы не учли это при первоначальной сборке, в результате чего на первом отборочном туре в городе Сочи при усиленных тренировках на эти рейки, которые на тот момент были установлены параллельны земле, оказывалось крайне большое давление в неправильной плоскасти, что вызвало поломку этих мебельных реек(подвижная часть так сильно отогнулась от неподвижной, в результате чего из рейки вылетели все подшипники, обеспечивающие работу рейки). Мы осознали, что мебельные рейки в производстве используются немного подругому, то есть их устанавливают в другой плоскости. Представим тумбочку с выдвижными ящиками. Её движение обоеспечиваюдт мебельные рейки схожие с теми, что использовали мы, только в тумбочке они установлены перпендикулырно земле, что не приводит к отгибу подвижной части рейки от неподвижной.
      \item Разметить места для сверления балок для обеспечения свободного прохождения реек через отверстия в балках.
      \item Укрепить ось, через которую перекидывается леска, так как из-за большой нагрузки на эту ось она деформировалась(прогнулась).
    \end{itemize}
    \item Результаты:
    \begin{itemize}
      \item Мебельные рейки установлены боком. Теперь пространство между сторонами будущего подъёмника значительно увеличилось по сравнению с предыдущей версией робота за счёт того, что мы установили мебельные рейки на несущие балки(на основу конструкции). Таким образом пространство в роботе у нас сильно увеличилось, что способствует созданию большей корзины и механизма для захвата.
      \item Отверстия для реек размечены, отверстия пока не просверленны, так как для этого необходимо проводить довольно масштабную разборку робота.
      \item Ось, через которую перекидывается леска, укреплена - на неё надет алюминиевый профиль, что сделало её вомного раз более крепкой и устойчивой к прогибу.
    \end{itemize} 
    \item Идеи:
    \begin{itemize}
    	\item Основной план - продолжать переделывать робота, в частности начать изготавливать подъёмник из уже распиленных алюминиевых балок.
	\item Необходимо грамотно установить проводку на нашего робота таким образом, чтобы был доступен блок NXT, была легкодоступна кнопка включения и выключения, чтобы можно было оперативно заменить аккумулятор NXT и TETRIX или подзарядить его. Необходимо, чтобы все контроллеры были подключены параллельно, так как при поломке одного из контроллеров, подключенных последовательно, остальные перестанут питаться электричеством, что может привести к полной демобилизации нашего робота.
    	\item Также необходимо сделать ёмкость для захвата шариков с захватом для них, её планируется сделать из пластика, так как пластик наиболее пластичен и с ним удобно работать.
    \end{itemize}
   \end{enumerate}
\fillpage
		