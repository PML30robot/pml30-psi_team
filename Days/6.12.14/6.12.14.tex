\documentclass[12pt]{article}
\usepackage{ucs}
\usepackage[utf8x]{inputenc} 
\usepackage[russian]{babel}  
\begin{document}
  \section{5.12.14}
  \begin{enumerate}
    \item Дата собрания : 24.09.14
  	\item Цель:
  	\begin{itemize}
  	  \item Рассверлить размеченные балки.
  	  \item Начать собирать один из нижних пролётов подъёмника. 
  	  \item Укрепить констуркцию, добавив спереди балку.
  	\end{itemize}
  	\item Результаты:
  	\begin{itemize}
  	  \item Балки рассверлены.
  	  \item Один из шести пролётов подъемника собран, при тестировании оказалось, что элементы подъёмника задевают за крепления балки, поэтому рассверленные балки переставлены.
  	  \item Спереди добавлены две разной толщины балки, что очень существенно добавило прочности конструкции при некоторых видах деформаций, при этом сохранена возможность прохождения больших шариков под робота спереди, запас составляет около 2 сантиметров. 
  	\end{itemize} 
  	\item Идеи:
  	\begin{itemize}
  	  \item На следующих занятиях необходимо продолжить сборку робота, в том числе подъёмника и захвата шариков.
  	\end{itemize}
  \end{enumerate}
\end{document}	