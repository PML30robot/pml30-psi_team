\begin{enumerate}
  	
  \item Дата собрания: 14.01.15
	
  \item Цели собрания: 
  \begin{itemize}
    \item Собрать корзину для мячей.
		
	\item 
  \end{itemize}

  \item Реализация:
  \begin{itemize}
    \item Была размечена и вырезана из ПЭТа развёртка корзины. 
	
	\item Корзина была склеена по швам, её форма, в отличие от предыдущей, - не прямоугольный параллелепипед, поскольку верхняя её грань наклонена, так что створ корзины идёт раструбом с увеличивающимся сечением. Это облегчает забрасывание шариков и обеспечивает его при меньшем угле поворота сервопривода.
    
    \item В обеих стенках корзины были вырезаны сплошные продольные отверстия в высоту маленького мяча, что обеспечивает их отсеивание. Это крйне выгодно с точки зрения игры.
    
    \item Для более точного закидывания мячей к верхней плоскости корзины были приклеены две направляющие, сходящиеся к центру до ширины большого мячика. Это необходимо, поскольку корзина в сечение имеет ширину вдвое больше, чем диаметр сечения подвижной корзины игрового поля.
  
  \end{itemize}
	
  \item Итоги собрания:
  \begin{itemize}
	\item Из ПЭТа была склеена коробка для корзин с отверстиями для отсеивания маленьких мячиков, наклонной верхней плоскостью для облегчения выкидывания шариков и направляющие для шариков, необходимые для центровки их сброса.
    
  \end{itemize}
	
  \item Задачи для последующих собраний:
  \begin{itemize}
    \item Необходимо сделать верхния металлический каркас для корзины и с помощью него закрепить её.
  \end{itemize}
\end{enumerate}
\fillpage