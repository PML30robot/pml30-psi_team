\begin{enumerate}
  	
  \item Дата собрания: 05.01.15
	
  \item Цели собрания: 
  \begin{itemize}
    \item Заменить ось, скрепляющую две стороны подъёмника балкой для лучшей фиксации двух сторон подъёмника друг относительно друга и уменьшения раскачивания подъёмника в разложенном состоянии.
		
	\item Отодвинуть омни колёса ближе к середине робота, для того, чтобы увеличить пространство для крепления откосов для шаров.
		
  \end{itemize}

  \item Реализация:
  \begin{itemize}
    \item Омниколёса были отодвинуты назад, что увеличило пространство для крепления откосов, но этого оказалось недостаточно, поэтому было режено оставить такие же откосы, несмотря на их ненадёжность.
		
	\item Ось была заменена балкой, но после этого возникла проблема с тем, как крепить леску к балке. При использовании старого способа крепления острые края балки при раскладывании подъёмника могли оказывать очень большую нагрузку на леску, кроме того, из-за того, что балка является в сечении прямоугольником, а не кругом, это воздействие распределялось бы не равномерно, а на четыре точки лески. Всё это приводило бы к быстрому износу лески. Поэтому было решено крепить леску по-другому.
		
    \item Крепление лески изменено: теперь на леске завязан узел, который одет на болт, вкрученный в балку. Две лески на двух болтах прижаты к балке листом оцинковки, одетой на болты и прикрученной гайками.
		
  \end{itemize}
	
  \item Итоги собрания:
  \begin{itemize}
	\item Омниколёса отодвинуты назад.
    
    \item Откосы для шариков остались такими же.
    
	\item Ось, скрепляющая стороны подъёмника заменена балкой.
	
	\item Крепление лески заменено на более подходящее при данной конструкции.
  \end{itemize}
	
  \item Задачи для последующих собраний:
  \begin{itemize}
    \item Укрепить ось, на которую наматывается леска.
  \end{itemize}
\end{enumerate}
\fillpage
