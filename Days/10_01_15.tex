\begin{enumerate}
  	
  \item Дата собрания: 10.01.15
	
  \item Цели собрания: 
  \begin{itemize}
    \item Укрепить балки, удерживающие ось, через которую перекинута леска. 
		
	\item Переделать захват колбы, чтобы нагрузка шла не на ось сервопривода, а на жёстко закреплённую шестерню.
  \end{itemize}

  \item Реализация:
  \begin{itemize}
    \item Балки, удерживающие ось были укреплены двумя другими балками, которые, к тому же, увеличили общую жёсткость рамы робота. 
    
    \item Они расположены так, что защищают саманту с одной стороны робота и контроллер сервоприводов с другой от ударов и столкновений с другими роботами.
	
	\item Захват колбы переделан так, что вся нагрузка, которая выламывала ось сервопривода теперь приходится на шестерёнку.
	
	\item Теперь между сервоприводом и самим захватом для колб существует понижающая передача 2:1. Это позволяет устранить возможность проворота сервопривода из-за нагрузки как во время игры, так и после отключения питания за счёт сил трения в самом сервоприводе.    
   
  \end{itemize}
	
  \item Итоги собрания:
  \begin{itemize}
	\item Балки, удерживающие ось, через которую перекинута леска, были усилены, что в целом укрепило раму и увеличило защищённость некоторых электронных компонентов.
    
    \item Захват для корзин был изменён для меньшей нагруженноти сервопривода.
    
	\item Благодаря новой конструкции захвата для корзин уменьшена вероятность непредвиденного поднятия захвата во время игры и после её окончания.
	
  \end{itemize}
	
  \item Задачи для последующих собраний:
  \begin{itemize}
    \item 
  \end{itemize}
\end{enumerate}
\fillpage