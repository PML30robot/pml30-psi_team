
  \begin{enumerate}
    \item Дата собрания : 5.12.14
    \item Цель:
    \begin{itemize}
      \item Установить мебельные рейки боком к земле для того, чтобы они выдерживали бОльшую нагрузку.
      \item Разметить места для сверления балок для обеспечения свободного прохождения реек через отверстия в балках.
      \item Укрепить ось, через которую перекидывается леска.
    \end{itemize}
    \item Результаты:
    \begin{itemize}
      \item Мебельные рейки установлены боком, теперь пространство между сторонами будущего подъёмника значительно увеличилось по сравнению с предыдущей версией робота.
      \item Отверстия для реек размечены, отверстия пока не просвернлены, так как для этого необходимо проводить довольно масштабную расборку робота.
      \item Ось, через которую перекидывается леска, укреплена - на неё надет алюминиевый профиль, что сделало её вомного раз более крепкой и устойчивой к прогибу.
    \end{itemize} 
    \item Идеи:
    \begin{itemize}
    	\item Как таковых идей нет, основной план - продолжать переделывать робота, в частности начать изготавливать подъёмник из уже распиленных алюминиевых балок.
    	\item Также необходимо сделать ёмкость для захвата шариков с захватом для них, её планируется сделать из пластика.
    \end{itemize}
   \end{enumerate} 
\fillpage