\documentclass[11pt]{article}
\usepackage{ucs}
\usepackage{graphicx}
\usepackage[utf8x]{inputenc} 
\usepackage[russian]{babel}  

\title{ Техническая книга, ФМЛ №30, команда ПСИ}

\begin{document}
	\maketitle
	\tableofcontents{}
	\newpage
	
	\section{Состав команды}
		\begin{table}[h]
			\begin{tabular}{|l|l|l|l|}
				\hline
				\textit{ФИО}         & Год рождения & Место учебы   & Роль в команде                                   \\ \hline
				Жадковский Александр &  1998            & ФМЛ №30       & Капитан команды                                  \\ \hline
				Лутошкин Роман       &  1998         & Гимназия №642 & Ответственный за техническую книгу, оператор № 1 \\ \hline
				Ильясов Александр    &  1999            & ФМЛ №30       & Оператор №2                                      \\ \hline
				Поникаровский Антон  & 1998         & ФМЛ №30       & Запасной оператор                                \\ \hline
			\end{tabular}
		\end{table}
	\section{Описание робота}
		\subsection{Конструкция}
			\begin{itemize}
				\item Робот должен быть небольшим и мобильным
				\item Конструкция должна быть наиболее простой с максимально легким доступом ко всем узлам конструкции
				\item Робот должен обладать механизмом подъема, способным подниматься на высоту 120 см и выше
				\item По возможности робот должен обладать специальным приспособлением для зацепки корзин и их перемещения
			\end{itemize}
		\subsection{Стратегия}
			Период выступления делится на 2(3) периода: автономный период и основное время, которое состоит из первых 1.5 минут и последних 30 секунд.
			В автономном периоде робот должен:
			\begin{itemize}
				\item В зависимости от расположения, съехать с пандуса или выехать вперед
				\item Сориентироваться согласно ИК-датчику и сбить подпорку корзины
				\item Захватить максимально возможное кол-во шариков(Но не более 5-ти)
			\end{itemize}
			После автономного периода следует управляемый двухминутный период в котором необходимо:
			\begin{itemize}
				\item Выгрузить захваченые в автономном периоде шарики в центр. корзину
				\item Захватить новые шарики 
				\item Повторять такую процедуру до окончания времени
				\item В конце вернуться на зону парковки
			\end{itemize}
		
	\section{Основная часть}
	
	\subsection{16.09.14}

	\subsection{3.10.14}

	
	\subsection{13.10.14}
	
	
\end{document}