\documentclass[12pt]{article}
\usepackage{ucs}
\usepackage{graphicx}
\usepackage[utf8x]{inputenc} 
\usepackage[russian]{babel}  
\title{ Техническая книга, ФМЛ №30, команда А}
\begin{document}
	\section{24.09.14}
	\begin{enumerate}
		\item Дата собрания : 24.09.14
		\item Цель:
		\begin{itemize}
			\item Перевернуть двигатели передвижения в креплениях для увеличения дорожного просвета.
			\item Укрепить и окончательно установить катушки для наматывания лески.
			\item Написать программу управления роботом с помощью стиков геймпада и протестировать движение робота, в том числе заезд и съезд с пандуса.
		\end{itemize}
		\item Результаты:
		\begin{itemize}
			\item Двигатели перевёрнуты, дорожный просвет увеличился примерно на 2 см.
			\item Катушки укреплены, барабан намотки увеличен, что позволяет наматывать леску с большей скоростью.
			\item Программа написана, робот заезжает на пандус достаточно долго и неуклюже.
			\item Немного изменена проводка, добавлена деталь, которой не хватало на прошлых занятиях – кнопка включения питания.
		\end{itemize}	
		\item Идеи:
		\begin{itemize}
			\item Заменить в подъёмнике рейки из набора TETRIX на алюминиевый профиль, соединить два подъёмника поперечными рейками, в т.ч. той, которую будет тянуть леска, намотать леску, проверить работу подъёмника.
			\item В будущем изменить колёсную базу, т.к. эта показывает плохие результаты при заезде на пандус.
		\end{itemize}
	\end{enumerate}
\end{document}	